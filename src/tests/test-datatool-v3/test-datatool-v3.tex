\documentclass{article}

\usepackage{datatool}

\DTLsetup{store-datum,
 %datetime=parse-only
 %datetime=reformat
}

\renewcommand{\DataToolTimeFmt}[3]{#1:#2}
\renewcommand{\DataToolDateFmt}[4]{#3/#2/#1}

\begin{document}
No value: \dtlnovalue. Data type: \DTLdatumtype{\dtlnovalue}.

\DTLparse\mydatum{2000}%\show\mydatum
String value: \mydatum. Numeric value: \DTLdatumvalue{\mydatum}.
Data type: \DTLdatumtype{\mydatum}.

% The Julian day number should be 2451545.25
\DTLparse\mydatum{2000-01-01 18:00:00}%\show\mydatum
String value: \mydatum. Numeric value: \DTLdatumvalue{\mydatum}.
{\renewcommand\DTLtemporalvalue[2]{#2}ISO: \DTLdatumvalue{\mydatum}}.
Data type: \DTLdatumtype{\mydatum}.

% The Julian day number should be 2451544.75 (2000-01-01 06:00 UT)
\DTLparse\mydatum{2000-01-01 07:00:00+01:00}%\show\mydatum
String value: \mydatum. Numeric value: \DTLdatumvalue{\mydatum}.
{\renewcommand\DTLtemporalvalue[2]{#2}ISO: \DTLdatumvalue{\mydatum}}.
Data type: \DTLdatumtype{\mydatum}.

\DTLsettemporaldatum{\mydatum}
 {29th Aug 2024 10:02 BST}
 {2024-08-29T10:02:20+01:00}
%\show\mydatum

String value: \mydatum. Numeric value: \DTLdatumvalue{\mydatum}.
{\renewcommand\DTLtemporalvalue[2]{#2}ISO: \DTLdatumvalue{\mydatum}}.
Data type: \DTLdatumtype{\mydatum}.

\DTLread[format=dtltex]{test-data-3}

Last loaded: \dtllastloadeddb.
Row count: \DTLrowcount{mydata}.
Column count: \DTLcolumncount{mydata}.

\DTLdisplaydb{mydata}

\DTLread[format=dbtex]{test-data-3}
%\DTLread[format=dbtex]{test-data-3expb}
Last loaded: \dtllastloadeddb.

\DTLdisplaydb{data3}

\DTLread[format=dbtex]{test-datum-3}
%\DTLread[format=dbtex]{test-datum-3expb}
Last loaded: \dtllastloadeddb.

\DTLdisplaydb{datum}

\DTLread[name=literal,csv-content=literal,format=csv]{test-literal}
Last loaded: \dtllastloadeddb.

\DTLdisplaydb{literal}

\DTLread[name=texdb,csv-content=tex,format=csv]{test-tex}
Last loaded: \dtllastloadeddb.

\DTLdisplaydb{texdb}

\providecommand{\qt}[1]{``#1''}
\DTLaction{new}
\DTLaction[key={Name}]{add column}
\DTLaction[key={Score},type=int]{add column}
\DTLaction[key={Award},value={Award (\$)}]{add column}
\DTLaction{new row}
\DTLaction[
 key={Name},
 value={Zoë Zebra}
]
 {new entry}
\DTLaction[key={Score},value={23}]{new entry}
\DTLaction
 [
   key={Award}, value={\$1,234}
 ]
 {new entry}
\DTLaction{new row}
\DTLaction
 [
   %key={Name},
   column=1,
   expand-value={Dickie \qt{the Duck}}
 ]
 {new entry}
\DTLaction[key={Score},value={34}]{new entry}
\DTLaction[key={Award},value={\$1,234.56}]{new entry}
\DTLaction
 [
   assign =
    {
      Name = {José Arara},
      Score = 68,
      Award = {\$2,453.99}
    }
 ]
 { new row }


\DTLaction{display}

Pad leading zeros: \dtlpadleadingzeros{3}{4},
\dtlpadleadingzeros{4}{1.5}, \dtlpadleadingzeros{5}{-.3},
\dtlpadleadingzeros{7}{12}, \dtlpadleadingzeros{2}{.8}.


\DTLread[format=dbtex,name=temporal1]{temporal1}

\DTLdisplaydb{temporal1}


\DTLread[format=dbtex,name=temporal2]{temporal2}

\DTLdisplaydb{temporal2}

\DTLsetup{numeric={auto-reformat}}

\DTLread[format=csv,name=plain,csv-content=no-parse,
  %data-types={decimal},
  data-types={integer,decimal,currency,decimal},
  only-reformat-columns={4}
 ]
{test-plain}

\DTLdisplaydb{plain}
\end{document}
